% Options for packages loaded elsewhere
\PassOptionsToPackage{unicode}{hyperref}
\PassOptionsToPackage{hyphens}{url}
%
\documentclass[
]{article}
\usepackage{amsmath,amssymb}
\usepackage{iftex}
\ifPDFTeX
  \usepackage[T1]{fontenc}
  \usepackage[utf8]{inputenc}
  \usepackage{textcomp} % provide euro and other symbols
\else % if luatex or xetex
  \usepackage{unicode-math} % this also loads fontspec
  \defaultfontfeatures{Scale=MatchLowercase}
  \defaultfontfeatures[\rmfamily]{Ligatures=TeX,Scale=1}
\fi
\usepackage{lmodern}
\ifPDFTeX\else
  % xetex/luatex font selection
\fi
% Use upquote if available, for straight quotes in verbatim environments
\IfFileExists{upquote.sty}{\usepackage{upquote}}{}
\IfFileExists{microtype.sty}{% use microtype if available
  \usepackage[]{microtype}
  \UseMicrotypeSet[protrusion]{basicmath} % disable protrusion for tt fonts
}{}
\makeatletter
\@ifundefined{KOMAClassName}{% if non-KOMA class
  \IfFileExists{parskip.sty}{%
    \usepackage{parskip}
  }{% else
    \setlength{\parindent}{0pt}
    \setlength{\parskip}{6pt plus 2pt minus 1pt}}
}{% if KOMA class
  \KOMAoptions{parskip=half}}
\makeatother
\usepackage{xcolor}
\usepackage[margin=1in]{geometry}
\usepackage{graphicx}
\makeatletter
\def\maxwidth{\ifdim\Gin@nat@width>\linewidth\linewidth\else\Gin@nat@width\fi}
\def\maxheight{\ifdim\Gin@nat@height>\textheight\textheight\else\Gin@nat@height\fi}
\makeatother
% Scale images if necessary, so that they will not overflow the page
% margins by default, and it is still possible to overwrite the defaults
% using explicit options in \includegraphics[width, height, ...]{}
\setkeys{Gin}{width=\maxwidth,height=\maxheight,keepaspectratio}
% Set default figure placement to htbp
\makeatletter
\def\fps@figure{htbp}
\makeatother
\setlength{\emergencystretch}{3em} % prevent overfull lines
\providecommand{\tightlist}{%
  \setlength{\itemsep}{0pt}\setlength{\parskip}{0pt}}
\setcounter{secnumdepth}{-\maxdimen} % remove section numbering
\usepackage{amsmath}
\usepackage{xcolor}
\usepackage{graphicx}

%%%%%%%%%%%%% defining colors
\definecolor{navyblue}{RGB}{18, 97, 158}

\usepackage{caption}
\usepackage{fancyhdr}
\usepackage[document]{ragged2e}
\pagestyle{fancy}

\fancyfoot[L]{Équipe 9} % NUMÉRO D'ÉQUIPE
\fancyfoot[R]{ACT-2011} % SIGLE DU COURS
\fancyfoot[C]{\thepage}

\setlength{\headheight}{12.80502pt}

\renewcommand{\contentsname}{Table des Matières}
\newcommand{\bcenter}{\begin{center}}
\newcommand{\ecenter}{\end{center}}
\ifLuaTeX
  \usepackage{selnolig}  % disable illegal ligatures
\fi
\usepackage{bookmark}
\IfFileExists{xurl.sty}{\usepackage{xurl}}{} % add URL line breaks if available
\urlstyle{same}
\hypersetup{
  hidelinks,
  pdfcreator={LaTeX via pandoc}}

\author{}
\date{\vspace{-2.5em}}

\begin{document}

%\documentclass[12pt]{article}
%\usepackage[english]{babel}
\begin{titlepage}

\newcommand{\HRule}{\rule{\linewidth}{0.5mm}} % Defines a new command for the horizontal lines, change thickness here

\center % Center everything on the page
\textsc{\LARGE Travail Pratique}\\[1.0cm]
\textsc{\Large Gestion du risque financier II}\\[0.2cm]
\textsc{\large ACT-2011}\\[0.7cm]
\textsc{\large Équipe 9}\\[0.7cm]

\HRule \\[0.4cm]
{ \Large \bfseries Rapport}\\[0.20cm] { \huge \bfseries Travail pratique}\\[0.20cm]

\HRule \\[2cm]

\begin{minipage}{0.4\textwidth}
    \begin{flushleft} \large
    \emph{Par}\\
        Lorélie Gélinas \textsc{}\\
        Charliane Larose \textsc{}\\
        Émie Leclerc \textsc{}
    \end{flushleft}
\end{minipage}%
\begin{minipage}{0.4\textwidth}
    \begin{flushright} \large
    \emph{Numéro d'identification}\\
        537 005 871\\
        XXX XXX XXX\\
        XXX XXX XXX\\
    \end{flushright}
\end{minipage} \\[1.0cm]

\emph{Travail présenté à} \\
\emph{Monsieur} \\[0.1cm]
\textsc{\Large Thai \textsc{Nguyen}}\\[1.0cm]

\textsc{\large 22 avril 2025}\\[1cm]
 
\vfill % Fill the rest of the page with whitespace

\end{titlepage}

\centering

\clearpage

\tableofcontents

\justify  
\clearpage

\section{Approximation des
paramètres}\label{approximation-des-paramuxe8tres}

Le premier paramètre à estimer est la volatilité. Celle-ci est définit
par

\[
\hat{\sigma} = \frac{Stdev(ln(S_{t+h}/S_t))}{{h^{1/2}}}.
\]

Comme les données fournies avec l'énoncé sont mensuelles, on a que
\(h = 1/12\). La valeur finale de \(\hat{\sigma}\) est 23.38\%.

La valeur du taux sans risque est estimée grâce à la moyenne des taux
d'intérêts effectifs annuels de chaque mois des cinq dernières années
(2019-03 au 2024-02). L'estimation du taux sans risque a pour valeur
\(r =\) 2.16\%. Comme le taux est utilisé sous forme continue dans les
formules d'arbre binomial, \(r\) a une valeur de 2.14\% de façon
continue.

Pour construire les arbres binomiaux, les valeurs de \(u\), \(d\) et
\(p\) sont nécessaires. Les formules suivantes permettent d'obtenir ces
valeurs

\[
u = e^{(r-\delta)h+\sigma\sqrt{h}},
\] \[
d = e^{(r-\delta)h-\sigma\sqrt{h}},
\] \[
p=\frac{e^{rh}-d}{u-d}.
\]

Comme l'énoncé mentionne une absence de dividende, on suppose que
\(\delta = 0\).

Pour l'arbre binomial à 4 périodes, on obtient que \(u =\) 1.13, \(d =\)
0.89 et \(p =\) 47.08\%.

Pour l'arbre binomial à 52 périodes, on obtient que \(u =\) 1.03,
\(d =\) 0.97 et \(p =\) 49.19\%.

\section{Arbres binomiaux}\label{arbres-binomiaux}

La présente section montre la démarche et les graphiques des arbres
binomiaux demandés. La fonction \texttt{binomopt} du paquetage
\texttt{derivmkts} a été grandement utilisée.

\subsection{Arbres binomiaux à 4
périodes}\label{arbres-binomiaux-uxe0-4-puxe9riodes}

Les fonctions \texttt{binomplot} et \texttt{binomopt} du paquetage
\texttt{derivmkts} permettent de construire les arbres binomiaux
demandés et d'en faire les graphes.

L'évolution du prix du sous-jacent pour l'option de vente avec 4
périodes avec un prix d'exercice de 95\$ est illustré ci-dessous.

\includegraphics{RapportGRF_equipe9_files/figure-latex/code_arbre_put95-1.pdf}

\begin{center}
\textsf{Figure 1 : Arbre binomial de l'option de vente européenne à 4 périodes}
\end{center}

On relève que l'option de vente européenne est levée pour seulement deux
prix du sous-jacent. Les points sont en vert. Les informations
pertinentes à l'arbre binomial sont soulevées directement sur la figure
ci-dessus.

\newpage

L'évolution du prix du sous-jacent pour l'option d'achat européenne avec
4 périodes avec un prix d'exercice de 110\$ est illustré ci-dessous.

\includegraphics{RapportGRF_equipe9_files/figure-latex/code_arbre_call110-1.pdf}

\begin{center}
\textsf{Figure 2 : Arbre binomial de l'option d'achat européenne à 52 périodes}
\end{center}

On observe que l'option de d'achat européenne est levée pour seulement
deux prix du sous-jacent. Les points où que l'option est levée sont en
vert. Les informations pertinentes à l'arbre binomial sont soulevées
directement sur le graphe ci-dessus. \newpage \#\# Arbres binomiaux à 52
périodes

Le prix pour l'option de vente européenne avec un prix d'exercice de
95\$, mais avec 52 périodes, est de 5.8938\$. Le prix de l'option
d'achat européenne avec prix d'exercice de 110\$ est 6.2972\$.

Les prix des options asiatiques ont été trouvés à l'aide de la fonction
\texttt{AsianOption} du paquetage \texttt{RQuantLib}. Le prix de
l'option d'achat arithmétique avec un prix d'exercice de 110\$ est
2.27\$. Le pris de l'option de vente à barrière désactivante de 105\$
est 2.73\$. Le prix d'exercice est 95\$ pour cette option.

\newpage

\section{Relation du prix de l'option et du prix
d'exercice}\label{relation-du-prix-de-loption-et-du-prix-dexercice}

\subsection{Option d'achat}\label{option-dachat}

On constate la relation du prix à payer pour l'option d'achat européenne
présentée à la section 2, celle avec le modèle 52 périodes, grâce à la
Figure 3 ci-dessous.

\begin{figure}
\centering
\includegraphics{RapportGRF_equipe9_files/figure-latex/graphique achat 52 periodes-1.pdf}
\caption{Graphique du prix d'une option d'achat européenne en fonction
du prix d'exercice}
\end{figure}

\begin{center}
\textsf{Figure 3 : Coût de l'option d'achat en fonction du prix d'exercice}
\end{center}

La Figure 3 illustre une relation inversement proportionnelle entre le
prix à payer pour l'option d'achat et le prix d'exercice. Ce
comportement est attendu : plus le prix d'exercice est faible, plus le
prix à payer pour l'option sera élevé, car il est plus avantageux
d'acheter l'actif sous-jacent à un prix inférieur que sa valeur
actuelle.Donc, il a une augmentation des probabilités que l'option soit
levée.\\
À l'inverse, un prix d'exercice élevé entrain une diminution du prix de
l'option puisqu'il est moins probable qu'elle soit exercée, car il est
moins avantageux d'acheter l'actif sous-jacent à un prix supérieur que
sa valeur actuelle.

\newpage

\subsection{Option de vente}\label{option-de-vente}

On constate la relation du prix à payer pour l'option de vente
européenne présentée à la section 2, celle avec le modèle 52 périodes,
grâce à la Figure 4 ci-dessous.

\begin{figure}
\centering
\includegraphics{RapportGRF_equipe9_files/figure-latex/graphique vente 52 periodes-1.pdf}
\caption{Graphique du prix d'une option de vente européenne en fonction
du prix d'exercice}
\end{figure}

\begin{center}
\textsf{Figure 4 : Coût de l'option de vente en fonction du prix d'exercice}
\end{center}

La Figure 4 illustre une relation directement proportionnelle entre le
prix à payer pour l'option de vente et le prix d'exercice. Ce
comportement est attendu : plus le prix d'exercice est élevé, plus il y
a de chance que l'option soit exercée. En effet, il est plus avantageux
de vendre l'actif à un prix supérieur que sa valeur actuelle, il y a
donc plus de chance que l'option de vente soit exercée lorsque le prix
d'exercice augmente.\\
À l'inverse, il est moins avantageux de vendre l'actif à un prix
inférieur que sa valeur actuelle. Donc, lorsque le prix d'exercice
diminue, le prix de l'option diminue, car il y a moins de chance qu'elle
soit exercée.

\newpage

\section{Arbre bionimal avec option vente
américaine}\label{arbre-bionimal-avec-option-vente-amuxe9ricaine}

Avec les mêmes fonctions utilisées pour construire les arbres binomiaux
présentés à la section 2, un arbre binomial 4 périodes pour une option
de vente américaine a été construit. La figure ci-dessous présente les
détails de l'arbre :

\includegraphics{RapportGRF_equipe9_files/figure-latex/code arbre put americain-1.pdf}

\begin{center}
\textsf{Figure 5 : Arbre binomial de l'option d'achat américaine à 4 périodes}
\end{center}

Les paramètres de l'arbre sont les suivants :

Une première différence entre l'option de vente américaine et l'option
de vente européenne (présentée à la section 2) est leur prix d'achat.
L'option de vente américaine a un prix d'achat plus élevée que l'option
de vente européenne, soit \textbf{afficher les valeurs}. Ceci est dû au
fait qu'il y a plus de chance que l'option de vente américaine soit
levée en raison de la possibilité d'exercer l'option avant l'échéance.
Cette différence est refletée dans chacun des noeuds. En effet, comme
les prix des options sont différents à chacun des noueds, les
portefeuilles réplicatifs sont aussi différents.

Cela introduit la deuxième différence, soit le nombre de possibilités où
les options sont levées. En effet, l'option de vente américaine est
levée hâtivemetn à \(0,8\) année. Il y a donc une seule option
d'exercice hâtif. Cependant, à l'échéance, les deux options sont
exercées pour les mêmes valeurs, soient \(80,9\) et \(64\).

Une ressemblance entre les deux options est représentée sur le haut du
graphique. En effet, aucune des options n'est exercée pour les valeurs
supérieures de l'actif, donc il n'y a aucun portefeuille réplicatif et
aucune possibilité d'exercice hâtif. Ainsi, lorsque le sous-jacent
augmente de valeur, aucune option n'est exercée.

\end{document}
